\documentclass{screenplay}
\title{The Flood}
\author{-}
\address{-}

\begin{document}
\coverpage

\fadein


\centretitle{14 ottobre, 2018}
\intslug[notte]{Soggiorno}
PADRE e FIGLIOMINORE di schiena sul divano illuminati dalla luce proveniente dallo schermo. 

\begin{dialogue}{telecomando}
Bip, bip.
\end{dialogue}

\begin{dialogue}[in crescendo]{Cronista}
Sembra che questa volta i tecnici dell'ARPAL e della Protezione civile siano riusciti a prevederla. L'ennesima 
alluvione si abbater\`{a} dunque sulla tartassata citt\`{a} di Genova. La previsione con sette giorni di anticipo 
permette alle autorit\`{a} e alla popolazione di organizzarsi e correre ai ripari.
\end{dialogue}

Si susseguono sullo schermo immagini drammatiche delle precedenti alluvioni.

\begin{dialogue}{Figliominore}
Mamma vieni a verdere!
\end{dialogue}

Di schiena compare MAMMA.

\begin{dialogue}{Cronista}
[..] dovrebbe quindi evitarsi la catastrofe del 2014 [..]
\end{dialogue}

\begin{dialogue}{---}
Porta blindata aperta e chiusa di scatto. Giaccavento, ombrello stropicciato e chiuso.
\end{dialogue}

\begin{dialogue}{FIGLIOMAGGIORE}
Pff! \textit{Belin}, non avete idea di che temporale c'\`{e} l\`{a} fuori.. E' strano che non sia ancora saltata la..
\end{dialogue}

Salta la luce.

(Attimi di attesa.)

\begin{dialogue}{---}
Prorompono le urla di donna, scroscio di acqua, onde, crolli, clacson, catastrofe.
\end{dialogue}

Un fulmine illumina con luce azzurrognola. Sequenza di spezzoni di alluvione e disastro.

Buio.


\centretitle{XX ottobre, 2018}
\begin{dialogue}{Voce gracchiante}
Avevano sbagliato. La catastrofe si \`{e} abbattuta su Genova con una settimana d'anticipo. Sono stati 6 giorni all'inferno. La 
citt\`{a} \`{e} in ginocchio. Tutto quello che era stato ricostruito ora non esiste pi\`{u}. E' una perdita immensa per l'Italia e una 
tragedia per l'umanit\`{a}. Il Presidente della Repubblica ha indetto il lutto nazionale. Il nostro supporto e cordoglio alla 
popolazione ligure. Si attende l'intervento dell'esercito. Messaggi di solidariet\`{a} arrivano da tutto il pianeta. 
\end{dialogue}


\centretitle{22 ottobre, 2018}
\extslug[mezzod\`{i}]{Campo profughi}
Fango, Tende, calca di gente, code per le coperte, coda per la cena, schiamazzi, pianti. Cordolo dell'esercito.


\centretitle{31 ottobre, 2018}
\extslug[tramonto]{Campo profughi}

Immagini di bambini in stracci che si rincorrono. Travestimenti di halloween di fortuna. Figura losca fa capolino 
dall'angolo di una tenda.


\centretitle{2 novembre, 2018}
\extslug[notte]{Ospedale da campo}
Via vai di barelle, casse di materiale medico. Ambulanza infangata e ammaccata. Personale in tuta di acrilico.

\begin{dialogue}{COPERTO DA ELICOTTERO}
Un'infezione ha colpito gli sfollati [..]. 
\end{dialogue}

\begin{dialogue}{OVATTATO SOTTO GASMASK}
Non possiamo fare nulla finch\'{e} non sapremo contro cosa stiamo combattendo [..] Restate in casa. Non bevete dai rubinetti [..]. Ripeto: rimanete in casa.
\end{dialogue}

\begin{dialogue}{}
DOON.
\end{dialogue}


\centretitle{8 dicembre, 2018}
\extslug[notte]{Zona bisagno}

Fasci di luce di torcia che si insinuano solidi tra cassonetti e macerie, nel vicolo. Pioggerellina sottile, mista 
a neve brilla nella luce.

\begin{dialogue}{OVATTATO SOTTO GASMASK}
Respiro regolare. Gorgoglio di una ricetrasmittente in cuffia.
\end{dialogue}

Gesti militari.

Un commando in tuta mimetica e visore notturno. Sono 5 uomini armati da incursione. Il respiro si condensa 
oltre la maschera.

\begin{dialogue}{---}
Scrapping, fruscio, tonfo su metallo.
\end{dialogue}

L'uomo malridotto che rovista nel cassonetto si accorge del commando. Sgrana gli occhi dal terrore quando gli spianano i mitra.

\begin{dialogue}{silenziatore}
RTT. RTT.
\end{dialogue}

L'uomo si accascia. 

Il commando si allontata. 

Cani randagi si avvicinano furtivi al cadavere. Primo piano di occhi rossi famelici, bava e alito caldo.

Chiarore di fuochi nell'alveo semi distrutto del Bisagno vicino alle arcate del ponte intasate di rifiuti trascinati 
dalla foga della corrente. Una palude di legname e rottami. Ombre che dipingono sul cemento. Graffiti e impronte insanguinate ballano con le fiamme. Uno di questi \`{e} una figura anfibia. Cthulhu.

Buio. 


\fadeout
Lentamente compare tremolando in primo piano il titolo di testa: ``THE FLOOD''

\theend

\end{document}

