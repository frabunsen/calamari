%   COMMANDS ZUM ZUSAMMENBAUEN DER KARTEN
%   ---------------------------------------


%%%%%%%%%%%%%%%%
% COLORICARTA  %
%%%%%%%%%%%%%%%%
\definecolor{titlebg}{RGB}{30,30,30}
\definecolor{contentbg}{RGB}{255,255,255}
\definecolor{pricebg}{RGB}{255,203,0}


%%%%%%%%%%%%%%%
% COLORIRUNE  %
%%%%%%%%%%%%%%%
\definecolor{abiurationrgb}{HTML}{7F3654}
\definecolor{conjurationrgb}{HTML}{547F36}
\definecolor{divinationrgb}{HTML}{54367F}
\definecolor{enchantmentrgb}{HTML}{367F54}
\definecolor{evocationrgb}{HTML}{7F5436}
\definecolor{illusionrgb}{HTML}{36547F}
\definecolor{necromancyrgb}{HTML}{000000}
\definecolor{transmutationrgb}{HTML}{990000}


%%%%%%%%%%%%%
% DINGRUNE  %
%%%%%%%%%%%%%
\newcommand{\base}{\ding{108}}
\newcommand{\sublime}{\ding{98}}
\newcommand{\upgrade}{\ding{164}}

%%%%%%%%%%%%%
% MOTECOST  %
%%%%%%%%%%%%%
\newcommand{\onemote}{\ding{202}}
\newcommand{\twomote}{\ding{203}}
\newcommand{\threemote}{\ding{204}}
\newcommand{\fourmote}{\ding{205}}
%\newcommand{\xmote}{\raisebox{.5pt}{\textcircled{\raisebox{-.9pt}{\small{\textbf{X}}}}}}
%\newcommand{\xmote}{{\large \textcircled{\small{X}}}}
\newcommand{\xmote}{\textcircled{\small{X}}}

%%%%%%%%%%%%%
% INTERLINE %
%%%%%%%%%%%%%
%\usepackage{setspace}
%\renewcommand{\baselinestretch}{.5} 
%%USAGE: {\setstretch{1.0} \scriptsize{#1}}


% TikZ/PGF Settings für die Karten
\pgfmathsetmacro{\cardwidth}{9} %%Original dimesniion is 6x9. This is still achievable, given you set only 1 line of content
\pgfmathsetmacro{\cardheight}{12}
\pgfmathsetmacro{\imagewidth}{\cardwidth}
\pgfmathsetmacro{\imageheight}{0.75*\cardheight}
\pgfmathsetmacro{\stripwidth}{0.7}
\pgfmathsetmacro{\strippadding}{0.2}
\pgfmathsetmacro{\textpadding}{0.1}
\pgfmathsetmacro{\titley}{\cardheight-\strippadding-1.5*\textpadding-0.5*\stripwidth}


%   Formen der einzelnen Kartenelemente/-bestandteile
\def\shapeCard{(0,0) rectangle (\cardwidth,\cardheight)}
\def\shapeLeftStripLong{(\strippadding,-0.2) rectangle (\strippadding+\stripwidth,\cardheight-\strippadding-\strippadding-1)}
\def\shapeLeftStripShort{(\strippadding,\cardheight-\strippadding-1) rectangle (\strippadding+\stripwidth,\cardheight+0.2)}
\def\shapeRightStripShort{(\cardwidth-\stripwidth-\strippadding,\cardheight-\strippadding-1) rectangle (\cardwidth-\strippadding,\cardheight+0.2)}
\def\shapeTitleArea{(2*\strippadding+\stripwidth,\cardheight-\strippadding) rectangle (\cardwidth-2*\strippadding-\stripwidth,\cardheight-2*\stripwidth)}
\def\shapeContentArea{(2*\strippadding+\stripwidth,0.5*\cardheight) rectangle (\cardwidth+0.2,-0.2)}
\def\shapeContentAreaGlog{(2*\strippadding+\stripwidth,0.75*\cardheight) rectangle (\cardwidth+0.2,-0.2)}

%   Stylings für die Elemente definieren
\tikzset{
    %   runde Ecken für die Karten
    cardcorners/.style={
        rounded corners=0.2cm
    },
    %   runde Ecken für die "Fähnchen"
    elementcorners/.style={
        rounded corners=0.1cm
    },
    %   Schlagschatten für die "Fähnchen"
    stripshadow/.style={
        drop shadow={
        	%shadow=true,
            opacity=0.5,
            color=black
        }
    },
    %   Style für die "Fähnchen"
    strip/.style={
        elementcorners,
        stripshadow
    },
    %   Bild für das Kartenmotiv
    cardimage/.style={
        path picture={
            \node[below=-1.5mm] at (0.5*\cardwidth,\cardheight) {
                \includegraphics[width=\imagewidth cm]{#1}
            };
        }
    },
}

%   TikZ-Raster
\newcommand{\carddebug}{padding
    \draw [step=1,help lines] (0,0) grid (\cardwidth,\cardheight);
}

%   Rahmen der Karte
\newcommand{\cardborder}{
    \draw[lightgray,cardcorners] \shapeCard;
}

%   Hintergrund der KarteTest
\newcommand{\cardbackground}[1]{
    \draw[cardcorners, cardimage=#1] \shapeCard;
}

%   Kategorie der Karte
\newcommand{\cardtype}[5]{
    %   First we fill the intersecting area
    %   The \clip command does not allow options, therefore 
    %   we have to use a scope to set th-0.2e even odd rule.
    \begin{scope}[even odd rule]
        %   Define a clipping path. All paths outside shapeCard will
        %   be cut because the even odd rule is set.
        \clip[cardcorners] \shapeCard;
        % Fill shapeLeftStripLong and shapeLeftStripShort.
        %   Since the even odd rule is set, only the card will be filled.
        \fill[strip,#1] \shapeLeftStripLong node[rotate=90,above left=0.9mm,font=\normalsize] {
            \color{white}\uppercase{#2}
        };
        \fill[strip,#1] \shapeLeftStripShort;
        
            
%    %%%%%%%%%%%%%% OPZIONALE: secondo cardtype 
    \ifthenelse{\equal{#4}{}}{}{
\def\shapeLeftStripMedium{(\strippadding,-0.2) rectangle (\strippadding+\stripwidth,\cardheight-\strippadding-\strippadding-4)}        
        \fill[strip,#4] \shapeLeftStripMedium node[rotate=90,above left=0.9mm,font=\normalsize] {
            \color{white}\uppercase{#5}
        };
      }
     %%%%%%%%%%%% FINE OPZIONALE: secondo cardtype 
    \end{scope}
    \node at (\strippadding+\stripwidth-0.28,\cardheight-\strippadding-\strippadding-0.37) {\color{white}#3};
}



%%%%%%%%%%%%%%%%
% COMMANDRUNE  %
%%%%%%%%%%%%%%%%
\newcommand{\cardtypeAbjuration}[1]{\cardtype{abiurationrgb}{Abiurazione}{\hspace{-1mm}\LARGE{#1}}{}{}}
\newcommand{\cardtypeConjuration}[1]{\cardtype{conjurationrgb}{Evocazione}{\hspace{-1mm}\LARGE{#1}}{}{}}
\newcommand{\cardtypeDivination}[1]{\cardtype{divinationrgb}{Divinazione}{\hspace{-1mm}\LARGE{#1}}{}{}}
\newcommand{\cardtypeEnchantment}[1]{\cardtype{enchantmentrgb}{Incantamento}{\hspace{-1mm}\\LARGE{#1}}{}{}}
\newcommand{\cardtypeEvocation}[1]{\cardtype{evocationrgb}{Invocazione}{\hspace{-1mm}\LARGE{#1}}{}{}}
\newcommand{\cardtypeIllusion}[1]{\cardtype{illusionrgb}{Illusione}{\hspace{-1mm}\LARGE{#1}}{}{}}
\newcommand{\cardtypeNecromancy}[1]{\cardtype{necromancyrgb}{Necromanzia}{\hspace{-1mm}\LARGE{#1}}{}{}}
\newcommand{\cardtypeTransmutation}[1]{\cardtype{transmutationrgb}{Trasmutazione}{\hspace{-1mm}\LARGE{#1}}{}{}}

\newcommand{\cardtypeOrthodox}[1]{\cardtype{abiurationrgb}{Mago di Gilda}{\hspace{-1mm}\LARGE{#1}}{}{}}
\newcommand{\cardtypeRedHand}[1]{\cardtype{evocationrgb}{Mago della Mano Rossa}{\hspace{-1mm}\LARGE{#1}}{}{}}
\newcommand{\cardtypeCleric}[1]{\cardtype{divinationrgb}{Chierico}{\hspace{-1mm}\LARGE{#1}}{}{}}
\newcommand{\cardtypeSword}[1]{\cardtype{conjurationrgb}{Mago delle Lame}{\hspace{-1mm}\LARGE{#1}}{}{}}
\newcommand{\cardtypeCulinary}[1]{\cardtype{illusionrgb}{Mago delle Lame}{\hspace{-1mm}\LARGE{#1}}{}{}}

%   Titel der Karte
\newcommand{\cardtitle}[1]{
    %\draw[pattern=soft crosshatch,rounded corners=0.1cm] \shapeTitleArea;
    \fill[elementcorners,titlebg,opacity=.85] \shapeTitleArea;
    \node[text width=3.75cm] at (0.5*\cardwidth,\titley) {
        \begin{center}
            \color{white}\uppercase{\normalsize #1}
        \end{center}
    };
}

%   Inhalt der Karte
\newcommand*{\cardcontent}[8]{
    \begin{scope}[even odd rule]
        \clip[cardcorners] \shapeCard;
        \fill[elementcorners,contentbg, opacity=0.7] \shapeContentArea;
    \end{scope}
    \node[below right, text width=(\cardwidth-2*\strippadding-\stripwidth-2*\textpadding-0.3)*1cm] (A) at (2*\strippadding+\stripwidth+\textpadding,0.5*\cardheight-\textpadding) {
        \begin{center}\small{
        \textbf{Tempo di Lancio}: {#1}\\
     	\textbf{Gittata}: {#2}\\
        \textbf{Componenti}: {#3}\\
        \textbf{Durata}: {#4}\\
        \textbf{Classi}: {#5}\\
        }
        \end{center}
        \vrule width \textwidth height 2pt \\[-2pt]\vspace{10pt}
        {#6}\\ \vspace{10pt}
        \ifthenelse{\equal{#7}{}}{}{\textbf{Ai Livelli Superiori}: {#7}\\}
    };
    \node[below right, text width=(\cardwidth-2*\strippadding-\stripwidth-2*\textpadding-0.3)*1cm] at 
    (2*\strippadding+\stripwidth+\textpadding,2) {
        \vrule width \textwidth height 2pt \\[-2pt] %% aggiunge un riga orizzontale spessa 2pt (col suo VSPACE	)
        \vspace{0.2cm}
        {\textit{\small{#8}}}
    };
}

\newcommand*{\cardcontentglog}[4]{
    \begin{scope}[even odd rule]
        \clip[cardcorners] \shapeCard;
        \fill[elementcorners,contentbg, opacity=0.5] \shapeContentAreaGlog;
    \end{scope}
    \node[below right, text width=(\cardwidth-2*\strippadding-\stripwidth-2*\textpadding-0.3)*1cm] (A) at (2*\strippadding+\stripwidth+\textpadding,0.75*\cardheight-\textpadding) {
        \begin{center}
        \small{
        	\textbf{R}: {#1}\\
     		\textbf{T}: {#2}\\
        	\textbf{D}: {#3}
        }
        \end{center}
        \vrule width \textwidth height 1pt \vspace{3pt}
        \small{#4}
    };
}


%   Preis der Karte
\newcommand{\cardprice}[1]{
    \begin{scope}[even odd rule]
        \clip[cardcorners] \shapeCard;
        \fill[strip,pricebg] \shapeRightStripShort;
    \end{scope}
    \node at (\cardwidth-0.5*\stripwidth-\strippadding,\titley-0.1) {\color{black} \textbf{#1}};
}


%%%\def\shapeMiaVerticale{(\cardwidth-\strippadding,-0.2) rectangle (\cardwidth-\strippadding-\stripwidth,\cardheight-\imageheight-5*\strippadding)}
%%%%	Card strenght/defence
%%%\newcommand{\cardpower}[1]{
%%%    \begin{scope}[even odd rule]
%%%        \clip[cardcorners] \shapeCard;
%%%        \fill[strip,pricebg] \shapeMia;
%%%    \end{scope}
%%%   \node at (\cardwidth-0.5*\stripwidth-\strippadding,\cardheight-\titley-0.1) {\color{black} #1};
%%%}


\def\shapeMia{(\cardwidth-2*\stripwidth,\strippadding+\stripwidth) rectangle (\cardwidth+0.2,1.5*\strippadding)}
%	Card strenght/defence
\newcommand{\cardpower}[1]{
    \begin{scope}[even odd rule]
        \clip[cardcorners] \shapeCard;
        \fill[strip,powerbg,opacity=1] \shapeMia;
    \end{scope}
    \node at (\cardwidth-0.6*\stripwidth-\strippadding,\cardheight-\titley-0.1) {\color{white} \textbf{#1}};
}

