\documentclass[a4paper,landscape]{article}


%%%%%%%%%%%%%%%
% GEOMETRIA	  %
%%%%%%%%%%%%%%%
\usepackage[margin=6mm,top=5mm]{geometry}

%%%%%%%%%%%%%%%
% LINGUA	  %
%%%%%%%%%%%%%%%
\usepackage[utf8]{inputenc}
%\usepackage[german]{babel}

%%%%%%%%%
% TESTO	%
%%%%%%%%%
\usepackage{microtype}
\usepackage{graphicx}
\usepackage{color}

%%%%%%%%%
% TIKZ	%
%%%%%%%%%
\usepackage{tikz}
\usetikzlibrary{patterns}
\usetikzlibrary{shadows}

\usepackage{ifthen}

%%%%%%%%%
% DING	%
%%%%%%%%%
\usepackage{pifont}


%   COMMANDS ZUM ZUSAMMENBAUEN DER KARTEN
%   ---------------------------------------


%%%%%%%%%%%
% COLORI  %
%%%%%%%%%%%
\definecolor{titlebg}{RGB}{30,30,30}
\definecolor{characterbg}{RGB}{0,100,200}
\definecolor{abilitybg}{RGB}{80,180,0}
\definecolor{itembg}{RGB}{200,50,50}
\definecolor{monsterbg}{RGB}{103,102,51}
\definecolor{testbg}{RGB}{180,50,150}
\definecolor{pricebg}{RGB}{230,180,0}
\definecolor{powerbg}{RGB}{30,30,30}
\definecolor{content}{RGB}{250,250,245}
\definecolor{contentbg}{RGB}{255,255,255}




%   TikZ/PGF Settings für die Karten
\pgfmathsetmacro{\cardwidth}{6}
\pgfmathsetmacro{\cardheight}{9}
\pgfmathsetmacro{\imagewidth}{\cardwidth}
\pgfmathsetmacro{\imageheight}{0.75*\cardheight}
\pgfmathsetmacro{\stripwidth}{0.7}
\pgfmathsetmacro{\strippadding}{0.2}
\pgfmathsetmacro{\textpadding}{0.1}
\pgfmathsetmacro{\titley}{\cardheight-\strippadding-1.5*\textpadding-0.5*\stripwidth}


%   Formen der einzelnen Kartenelemente/-bestandteile
\def\shapeCard{(0,0) rectangle (\cardwidth,\cardheight)}
\def\shapeLeftStripLong{(\strippadding,-0.2) rectangle (\strippadding+\stripwidth,\cardheight-\strippadding-\strippadding-1)}
\def\shapeLeftStripShort{(\strippadding,\cardheight-\strippadding-1) rectangle (\strippadding+\stripwidth,\cardheight+0.2)}
\def\shapeRightStripShort{(\cardwidth-\stripwidth-\strippadding,\cardheight-\strippadding-1) rectangle (\cardwidth-\strippadding,\cardheight+0.2)}
\def\shapeTitleArea{(2*\strippadding+\stripwidth,\cardheight-\strippadding) rectangle (\cardwidth-2*\strippadding-\stripwidth,\cardheight-2*\stripwidth)}
\def\shapeContentArea{(2*\strippadding+\stripwidth,0.5*\cardheight) rectangle (\cardwidth+0.2,-0.2)}


%   Stylings für die Elemente definieren
\tikzset{
    %   runde Ecken für die Karten
    cardcorners/.style={
        rounded corners=0.2cm
    },
    %   runde Ecken für die "Fähnchen"
    elementcorners/.style={
        rounded corners=0.1cm
    },
    %   Schlagschatten für die "Fähnchen"
    stripshadow/.style={
        drop shadow={
        	%shadow=true,
            opacity=0.5,
            color=black
        }
    },
    %   Style für die "Fähnchen"
    strip/.style={
        elementcorners,
        stripshadow
    },
    %   Bild für das Kartenmotiv
    cardimage/.style={
        path picture={
            \node[below=-1.5mm] at (0.5*\cardwidth,\cardheight) {
                \includegraphics[width=\imagewidth cm]{#1}
            };
        }
    },
}

%   TikZ-Raster
\newcommand{\carddebug}{padding
    \draw [step=1,help lines] (0,0) grid (\cardwidth,\cardheight);
}

%   Rahmen der Karte
\newcommand{\cardborder}{
    \draw[lightgray,cardcorners] \shapeCard;
}

%   Hintergrund der Karte
\newcommand{\cardbackground}[1]{
    \draw[cardcorners, cardimage=#1] \shapeCard;
}

%   Kategorie der Karte
\newcommand{\cardtype}[5]{
    %   First we fill the intersecting area
    %   The \clip command does not allow options, therefore 
    %   we have to use a scope to set th-0.2e even odd rule.
    \begin{scope}[even odd rule]
        %   Define a clipping path. All paths outside shapeCard will
        %   be cut because the even odd rule is set.
        \clip[cardcorners] \shapeCard;
        % Fill shapeLeftStripLong and shapeLeftStripShort.
        %   Since the even odd rule is set, only the card will be filled.
        \fill[strip,#1] \shapeLeftStripLong node[rotate=90,above left=0.9mm,font=\normalsize] {
            \color{white}\uppercase{#2}
        };
        \fill[strip,#1] \shapeLeftStripShort;
        
            
%    %%%%%%%%%%%%%% OPZIONALE: secondo cardtype 
    \ifthenelse{\equal{#4}{}}{}{
\def\shapeLeftStripMedium{(\strippadding,-0.2) rectangle (\strippadding+\stripwidth,\cardheight-\strippadding-\strippadding-4)}        
        \fill[strip,#4] \shapeLeftStripMedium node[rotate=90,above left=0.9mm,font=\normalsize] {
            \color{white}\uppercase{#5}
        };
      }
     %%%%%%%%%%%% FINE OPZIONALE: secondo cardtype 
    \end{scope}
    \node at (\strippadding+\stripwidth-0.28,\cardheight-\strippadding-\strippadding-0.37) {\color{white}#3};
}
\newcommand{\cardtypeCharacter}{\cardtype{characterbg}{Character}{\hspace{-1mm}\LARGE\ding{95}}{}{}}
\newcommand{\cardtypeAbility}{\cardtype{abilitybg}{Ability}{\hspace{-1mm}\Large\ding{97}}{}{}}
\newcommand{\cardtypeItem}{\cardtype{itembg}{Item}{\hspace{-1mm}\LARGE\ding{98}}{}{}}
\newcommand{\cardtypeMonster}{\cardtype{monsterbg}{Monster}{\hspace{-1mm}\LARGE\ding{90}}{}{}}
\newcommand{\cardtypeTest}{\cardtype{testbg}{Testcard}{\hspace{-1.4mm}\huge\ding{78}}{red}{Item}}

%   Titel der Karte
\newcommand{\cardtitle}[1]{
    %\draw[pattern=soft crosshatch,rounded corners=0.1cm] \shapeTitleArea;
    \fill[elementcorners,titlebg,opacity=.85] \shapeTitleArea;
    \node[text width=3.75cm] at (0.5*\cardwidth,\titley) {
        \begin{center}
            \color{white}\uppercase{\normalsize #1}
        \end{center}
    };
}

%   Inhalt der Karte
\newcommand*{\cardcontent}[2]{
    \begin{scope}[even odd rule]
        \clip[cardcorners] \shapeCard;
        \fill[elementcorners,contentbg, opacity=0.7] \shapeContentArea;
    \end{scope}
    \node[below right, text width=(\cardwidth-2*\strippadding-\stripwidth-2*\textpadding-0.3)*1cm] at (2*\strippadding+\stripwidth+\textpadding,0.5*\cardheight-\textpadding) {
        %\textit{\glqq\normalsize #1\grqq}
        \textit{#1}
    };
    \node[below right, text width=(\cardwidth-2*\strippadding-\stripwidth-2*\textpadding-0.3)*1cm] at (2*\strippadding+\stripwidth+\textpadding,3) {
        \vrule width \textwidth height 2pt \\[-2pt]
        \vspace{0.2cm}
        {\scriptsize #2}
    };
}


%   Preis der Karte
\newcommand{\cardprice}[1]{
    \begin{scope}[even odd rule]
        \clip[cardcorners] \shapeCard;
        \fill[strip,pricebg] \shapeRightStripShort;
    \end{scope}
    \node at (\cardwidth-0.5*\stripwidth-\strippadding,\titley-0.1) {\color{black} \textbf{#1}};
}


%%%\def\shapeMiaVerticale{(\cardwidth-\strippadding,-0.2) rectangle (\cardwidth-\strippadding-\stripwidth,\cardheight-\imageheight-5*\strippadding)}
%%%%	Card strenght/defence
%%%\newcommand{\cardpower}[1]{
%%%    \begin{scope}[even odd rule]
%%%        \clip[cardcorners] \shapeCard;
%%%        \fill[strip,pricebg] \shapeMia;
%%%    \end{scope}
%%%   \node at (\cardwidth-0.5*\stripwidth-\strippadding,\cardheight-\titley-0.1) {\color{black} #1};
%%%}


\def\shapeMia{(\cardwidth-2*\stripwidth,\strippadding+\stripwidth) rectangle (\cardwidth+0.2,1.5*\strippadding)}
%	Card strenght/defence
\newcommand{\cardpower}[1]{
    \begin{scope}[even odd rule]
        \clip[cardcorners] \shapeCard;
        \fill[strip,powerbg,opacity=1] \shapeMia;
    \end{scope}
    \node at (\cardwidth-0.6*\stripwidth-\strippadding,\cardheight-\titley-0.1) {\color{white} \textbf{#1}};
}



%   Dummy-Contents
\newcommand{\contentA}{Neunmalkluges Zitat\\mit Bezug zur Karte von einer fiktiven Figur.}
\newcommand{\contentB}{Auswirkungen:\\[5pt]Ziehe 2 Karten, nimm 3 Gold, lege alles ab, lauf weg …}

\begin{document}
\begin{center}
    \pagestyle{empty}

%   V-Space-Korrektur bei langen Titeln
%   \cardtitle{\vspace{-5mm}SEHR LANGER TITEL}
%%
    \begin{tikzpicture}
        \cardbackground{/home/sensi/Pictures/Fantasy/Characters/jason-l-carr.jpg}
        \cardtypeItem
        \cardtitle{Tombarolo}
        \cardcontent{C'è polvere quà dentro}{\contentB}
        \cardprice{5}
        \cardpower{0/1}
        \cardborder
    \end{tikzpicture}
    \hspace{5mm}
    \begin{tikzpicture}
        \cardbackground{/home/sensi/Pictures/SteamPunk-StarWars.jpg}
        \cardtypeAbility
        \cardtitle{Der Wächter}
        \cardcontent{Keine Sorge, ich mache es lang und äußerst schmerzhaft für dich.}{\contentB}
        \cardprice{35}
		\cardpower{4/1}
        \cardborder
    \end{tikzpicture}
    \hspace{5mm}
    \begin{tikzpicture}
        \cardbackground{/home/sensi/Pictures/Magicka_Vietnam_Poster_JPG.jpg}
        \cardtypeCharacter
        \cardtitle{Mächtiger Schlächter}
        \cardcontent{Eye, Cäpt'n. Diese Landradd'n werden wir über die Planken schicken!}{\contentB}
        \cardprice{435}
        \cardpower{2/2}
        \cardborder
    \end{tikzpicture}
    \hspace{5mm}
    \begin{tikzpicture}
        \cardbackground{/home/sensi/Pictures/Fantasy/Characters/alexander-nanitchkov.jpg}
        \cardtypeTest
        \cardtitle{Schlechter Wächter}
        \cardcontent{Sei Gast in meinem bescheidenen Domizil.}{\contentB}
        \cardprice{$\infty$}
        \cardpower{1/3}
        \cardborder
    \end{tikzpicture}

    \vspace{5mm}
    \begin{tikzpicture}
        \cardbackground{/home/sensi/Pictures/Fantasy/Characters/c-j-bloomer.jpg}
        \cardtypeItem
        \cardtitle{Schmächtiger Wächter}
        \cardcontent{Miau.}{\contentB}
        \cardprice{5}
        \cardpower{1/2}
        \cardborder
    \end{tikzpicture}
    \hspace{5mm}
    \begin{tikzpicture}
        \cardbackground{/home/sensi/Pictures/Fantasy/Characters/david-chen.jpg}
        \cardtypeAbility
        \cardtitle{Gebrechlicher Schlächter}
        \cardcontent{\contentA}{\contentB}
        \cardprice{35}
        \cardpower{1/1}
        \cardborder
    \end{tikzpicture}
    \hspace{5mm}
    \begin{tikzpicture}
        \cardbackground{/home/sensi/Pictures/Fantasy/Characters/Snake-queen.jpg}
        \cardtypeMonster
        \cardtitle{Gerechter Wächter}
        \cardcontent{\contentA}{\contentB}
        \cardprice{435}
        \cardpower{3/3}
        \cardborder
    \end{tikzpicture}
    \hspace{5mm}
        \begin{tikzpicture}
        \cardbackground{/home/sensi/Pictures/Fantasy/Characters/michael_whelan__stormbringer.jpg}
        \cardtypeCharacter
        \cardtitle{Elric of Melnibon\'{e}}
        \cardcontent{We would all be better off dead, useless eaters of the lotus that we are}{\contentB}
        \cardprice{999}
        \cardpower{3/6}
        \cardborder
    \end{tikzpicture}
    \hspace{5mm}
    
%%%% Test-card    
%%%    \begin{tikzpicture}
%%%        \carddebug
%%%        \cardtypeTest
%%%        \cardtitle{Welcher Schlächter?}
%%%        \cardcontent{\contentA}{\contentB}
%%%        \cardprice{$\infty$}
%%%        \cardborder
%%%        \cardpower{18/18}
%%%    \end{tikzpicture}   
    
\end{center}
\end{document}
