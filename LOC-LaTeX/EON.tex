\documentclass[10pt,a4paper]{report}
\usepackage[italian]{babel}
\usepackage[utf8x]{inputenc}
\usepackage{graphicx}
\usepackage{latexsym}

\begin{document}
\paragraph{Basi del gioco} 
\begin{enumerate}
\item{Tu} sei il Giocatore
\item Il \textbf{DM} è il Demiurgo del mondo di gioco
\item Il tuo alter-ego nel gioco è il personaggio \textbf{PG}
\item Il Giocatore sarà libero di far fare al suo \textbf{PG} qualunque azione deisderi. Se sarà destino che l'azione abbia successo, questa riuscirà, altrimenti fallirà
\item Il \textbf{Destino} nel gioco è rappresentato dal lancio dei dadi
\subitem Il lancio di dadi più comune è il lancio di 3d6$\star$
\item Il PG è descritto da 5 \textbf{Caratteristiche}
\subitem Le carattertiche hanno un punteggio di 1d6+10
\item Ciò che il PG sa fare è rappresentato dalle sue \textbf{Abilità}
\subitem Ogni Abilità ha una o più Caratteristriche di riferimento
\subitem Le Abilità hanno un punteggio determinato dalla Caratteristica di riferimento più un certo modificatore
\item Quando il PG vuole tentare un'azione esegue un \textbf{Tiro di Abilità}
\subitem Per riuscire in un Tiro di Abilità il risultato deve essere uguale o inferiore al suo punteggio di Abilità
\subitem Quando a contrapporsi all'azione di un PG non è il solo destino, ma un avversario, il PG dovrà ottenre un risultato uguale o inferiore al suo punteggio di Abilità, ma superiore al risultato ottenuto dall'avversario
\item In condizioni particolarmente avverse o favorevoli il tiro dei dadi può essere modificato a discrezione del \textbf{DM} rendendo più o meno semplice ottenere un successo nel Tiro di Abilità
\subitem Il \textbf{DM} può decidere che sia necessario aggiungere/sottrarre un numero al Tiro di Abilità
\subitem Il \textbf{DM} può decidere che sia necessario aggiungere/sottrarre un dado al Tiro di Abilità
\item Ai fini del gioco il tempo è scandito in \textbf{Round}
\subitem In ogni \textbf{Round} un \textbf{PG} ha a disposizione un certo numero di \textbf{PM} che limitano il numero di azioni che può compiere


\end{enumerate}



\end{document}
