\documentclass[10pt,a4paper]{article}
\usepackage[latin1]{inputenc}
\usepackage[italian]{babel}

\usepackage{fancybox}
\author{Frà}


\begin{document}

\section*{Domains}
\subsection*{Greater Domains} They are some of the more powerful domains available. They include the basic elements, concepts that are essential to civilization, and more abstract things like Time and Void. Remember, with great power comes great responsibility.

\subsection*{Moderate Domains} These are more specific than greater domains allowing for wider control over a certain aspect of existence such as weather and dragons.

\subsection*{Lesser Domains} These are the most specialized types of domains in the game, allowing the god or goddess who holds it unrivaled influence over that particular aspect of reality.
\subsubsection*{Examples}
A list of example domains of all three levels, for reference. These are by no means the only domains available, nor must these domains actually be present in the game.

\paragraph*{Greater Domains}
Air, Balance, Chaos, Creation, Darkness, Death, Destruction, Earth, Evil, Fate, Fire, Good, Harmony, Ice, Ki, Knowledge, Law, Life, Light, Magic, Nature, Order, Peace, Secrets, Time, Void, War, Water.

\paragraph*{Moderate Domains}
Aberrations, Animals, Bravery, Civilization, Conquest, Corruption, Decay, Diseases, Dragons, Dreams, Fey, Gravity, Healing, Honor, Invention, Justice, Language, Luck, Madness, Mercy, Metal, Moon, Music, Necromancy, Ocean, Passion, Pestilence, Plant, Protection, Psionics, Purification, Rebirth, Renewal, Retribution, River, Sky, Slaughter, Stars, Storm, Strategy, Strength, Sun, Travel, Trickery, Truth, Undead, Weather.

\paragraph*{Lesser Domains}
Athletics, Blacksmiths, Blue Dragons, Caverns, Chromatic Dragons, Cities, Courage, Day, Deception, Dwarves, Elves, Endurance, Envy, Evil Dragons, Fear, Fighters, Force, Forest, Gold Dragons, Greed, Halflings, Harvest, Hatred, History, Illusion, Insight, Joy, Kobolds, Lightning, Love, Lust, Machinery, Meditation, Mountains, Night, Pain, Pleasure, Poison, Pride, Revelry, Rogues, Runes, Scrying, Sea, Sight, Sleep, Song, Stealth, Suffering, Teleportation, Trade, Traps, Trees, Tyrants, Vanity, Vengeance, Victory, Wind, Wizards, Wrath.



\section*{Power Points}
Power Points (PP) are what you use to perform divine acts, like making mountains or creating monsters. PP comes from two sources: your gods base PP level and your attributes.

Every god gets a few base PP, based on their divine rank. These can be used towards any and all actions. A god get 1 PP/turn.



\subsection*{Actions}
The following is a list of the divine actions that a deity can perform and their base cost. All actions can be modified to have an increased or decreased effect with an associated increased or decreased cost. Examples are given when the different tiers of power may not be clear.

\paragraph*{Nourish (1 PP)}  This promotes life. It can turn a desert into a forest or produce basic flora and fauna. At higher levels it either has a wider area of effect or it can create valuable resources (see Resources, below).

\paragraph*{Harm (1 PP)}  This creates disasters, like tornado or plagues or it can make environments harsher. It can also initiate divine conflict.

\paragraph*{Command (2PP)}  This bends part of the world to your will, thereby raising a mountain range, or commanding a great forest to uproot itself and travel across a continent. Different levels cover different sized regions.

\paragraph*{Guide (2PP)}  Guide actions subtly influences the behaviors of mortals. This can cause them to adopt a mundane technology, learn a new character class, or go to war. At a lesser level (1PP) it can influence benign cultural traits, like their popular entertainment. At greater levels (3PP) it can grant fantastical technologies, like flying ships (see below for details on technologies).

\paragraph*{Spawn (2PP)}  This action produces animals and monsters that provide a challenge to mortals, and which are relatively uncommon throughout the world. Lower levels produce more common but weaker beasts, while higher levels create rarer but more powerful ones.

\paragraph*{Shape (3PP)}  This creates a sentient race that is native to the plane and environment it was created in. For example, Angels created in the prime material plane are, for the purposes of this setting, normal, mundane creatures. At this level, races are relatively average. At lower levels, they tend towards having larger populations but are individually weaker, and vice versa at higher levels.

\paragraph*{Beget (3PP)}  This gives birth to a new deity. This deity will be the character for a new player. The deity's domain is taken from your own, free and pre-existing and availible for claiming, or is created on the spot (a craft action is not required). At lower levels it can produce Exarchs (lesser characters for you to play) or legendary mortal heroes neither of which need domains.

\paragraph*{Forge (2PP)}  This action creates something that is part mundane and part divine, such as an artifact that shrouds a continent in an impassible storm, or a small demiplane defined and controlled by its creator. At a lower level, this creates legendary mortal artifacts. At a higher level it can produce full planes or powerful artifacts that can affect the gods.

\paragraph*{Craft (2PP)}  Creates or formalizes a concept fundamental to the universe. These can be claimed as domains, or they can be free floating concepts. At this level, the action produces fundamentals domains, like Fire. Lower levels produce more specific domains, while higher levels produce more generalized ones.


\section*{Resources and Technologies}
Mortal races flourish based on the resources available to them and the technology to properly harness those resources.  

While all populaces can utilize the technology of their day and age, some nations are known for their handiwork. If a god performs 2PP guide action, the race they bestow their blessing on is now an expert at that skill. Fantastical technologies, like creating airships, is a 3PP guide action.

Likewise, all races have some access to the resources necessary to go about their daily lives. A god may create a particularly rich resource, however, using a 2PP Nourish action (or a 3PP one for "fantastic" resources like mithril). All resources must be tied to a specific point in the world (Mithral might be only found in the mines of Moria, for example).

Both 2+ PP techs and 2+ PP resources give the civilization a +1 bonus (each) to civilization strength (see the Combat section for more information). If a tech and a resource are paired together (iron-working with an iron deposit, for instance), an extra +1 Synergy bonus is added to the civilization strength as well. If a fantastical tech and resource are matched, it grants a +2 Synergy bonus.
 
\section*{Combat}
Divine combat occurs when two or more gods come into conflict. This can be a physical confrontation, a verbal debate, a trick, etc. Ideally, combat should be planned in the OOC, but if the conflict is necessary and players can't agree, then there are rules to arbitrate the matter.

\begin{itemize}


\item God A and god B, want to duel
\item God A bets X AP, if god B wants to duel he will have to bet the same amount. Both have -X PP in their PP pool at the moment
\item Both gods roll: 1d20+ any combat artifacts + Divine Power
\item God B wins! However both sides still have -X PP from betting
\item God B can dispose of the \textit{victus} at will (see below)
\end{itemize}

Here are a few suggestions for what might occur when a god wins/loses a conflict (some of these aren't appropriate for a verbal debate, of course).
\textbf{Trophy}: Claim ownership of one artifact, or force into your service one exarch (still RP'ed by the original owner).\\
\textbf{Domain}: Gain dual-ownership of one domain from the loser (or, with original player's permission, sole ownership).\\
\textbf{Inhibit}: Prevent the loser from performing a specific sort of action (action cost increased by 1 for the loser).\\
\textbf{Imprison}: Confine the loser to a specific location, plane, etc. (does not prevent the prisoner from using PP).\\
\textbf{Maime}: Alter the loser's physical appearance through violence (take their eye, hand, etc).\\
\textbf{Liberation}: reclaim a lost artifact, exarch, domain, break an inhibition or a prison, or undo a maime.

Likewise, conflict can occur between two civilizations. This might be a war, a trade dispute, or cultural subversion. Rules arbitrating these conflicts are also found below. Note: You can only attack a specific nation once per turn. Attacking multiple turns in a row (regardless of target) results in war weariness, which increases the guide PP cost to start conflict by 1 (cumulative). War weariness recedes at the rate it accumulates.

Here are some suggestions for what might occur if a nation bests another nation (again, not all appropriate for all occurances)
\textbf{Tribute}: Receive the benefit from one resource that the losing nation possesses.\\
\textbf{Annex}: Take a chunk of land from the losing nation (size generally should scale with how much you won by) or convert the population there in to a particular belief system.\\
\textbf{Unrest}: Cause the collapse of the current government in the losing nation.\\
\textbf{Knowledge}: Steal a tech from the losing nation.\\
\textbf{Liberation}: Remove the oppression of past wars (stops tribute, regains annex, stills unrest).



\section*{Lords of Creation: Chaos Within}

Gods use Action Points (PP) to influence the world. Every week of real time, each player gains a number of PP proportionate to his god's power and level of influence; gods with many great works and armies of worshippers will gain PP faster, as will those who conduct great roleplaying and help influence the world. Your weekly PP starts at 4, and-if you've been playing well-can be as high as 9. PP is assigned by the game's admins.  You have until the end of sunday each week to use your PP from the prior week, otherwise it is lost.

You can spend PP to conduct various actions. Some example divine actions, and their corresponding PP costs, are as follows:

\subsubsection*{Command}
Command is used to issue divine orders. This can be as simple as ordering a hill to form to the creation of a volcanic mountain range. Or it could be used to order your worshippers to war against another faction or to teach people how to perform special actions. The exact potencty of ther action is determined by the amount of PP that is spend.\\
\textbf{Command Land (1 PP)} Glykllivar smote her spear into the ocean floor, causing a volcano to erupt from the spot. The lava flowed and the hardened, creating an island.\\
\textbf{Command Populace (1 PP)} The Sage appeared to the elven shamans in their dreams, teaching them the making of certain artifacts. When they awoke, they began scribing their magic on parchment, to make spellbooks and scrolls; the first Wizards had appeared.\\
\textbf{Command Land (2 PP)} Kur-ah strode along the center of the world, magma dripping from his body. Where it hardened, mountains formed, creating a new chain of hills and cliffs going across the continent.\\
\textbf{Command Populace (2 PP)} Morintu sent his exarch to preach to the primitive humans. Soon, these people were worshippers of Morintu, and sent armies to the nearby elven lands to conquer them in his name.\\
\textbf{Command Land (3 PP)} Kyrinian stole a handful of fire from Kur-ah's volcano, throwing it into the sky to create the sun.

\subsubsection*{Populate/Depopulate}
Populate is used to create life of all kinds in the world. \\
\textbf{Populate Birds (1PP)} Mariana lifted a handful of sand and threw it into the air. From the cascade, birds of many hues and types flew forth into thhe world.\\
\textbf{Populate Pegasus (2 PP)} From her divine realm, Artheyan looked down upon the wild horses that dwelt in her green fields. She raised her hand skyward, and several of the horses darted upward, using their new, feathery wings to fly.\\
\textbf{Populate Populace (3 PP)} When Britmara bathed herself in the cool lake, her power and wisdom took form amid the water. When she left the pool, so too did a newborn race; the werewolves.

Depopulate is an action that is rarely appropriate, and represents the complete genocide of a species, and is often not possible until all nourishes and other benefits have been stripped from that group.  Please confer with the admins if you wish to use this action. 


\subsubsection*{Nourish/Harm}
Used to help/hurt things.  \\
\textbf{Nourish Land (1 PP)} Wherever Artheyan's tears of grief fell upon the land, grasses and shrubs sprang up, bringing new life to this formerly barren land.\\
\textbf{Nourish Populace (1 PP)} The prayers of the villagers were answered that year; their newly planted fields flourished beyond measure. Their population and wealth began to increase, allowing them to expand northward and colonize new lands.\\
\textbf{Harm Land (1 PP)}  Pollux cried out, and the forest withered and burned under his fiery gaze.\\
\textbf{Harm Populace (1 PP)}  The rampaging armies of lizardmen torched the fields of the halflings, and famine fell upon them.

\subsubsection*{Create/Destroy}
\textbf{Create Artifact (1PP)} Tzu handed the warrior a silver spear, a relic that would be passed down for generations.\\
\textbf{Create Plane (2 PP)} Ooulzoth took one of his aberrant beasts and made it grow to a colossal size. The creature was now big enough to house entire armies within its body, and Ooulzoth built his palace within its stomach.\\
\textbf{Create Domain (3 PP)} Lexius drew a complex diagram across the map, using preexisting landmarks like islands and rivers as part of his immense sigil. As he completed the diagram, power pulsed through the bones of the world, and arcane magic was unleashed freely for anyone to use.\\
\textbf{Create Exarch (3 PP)} Morintu took the mortal Icoscol into the sky, and allowed him to overlook the mortal world from above. ``This can all be yours, someday''.\\
\textbf{Create Artifact (3 PP)} Kakavindus sat down in his dark palace, and began creating an ebony mask. He poured all of his cunning, malice, and lust for power into the artifact, imbuing it with terrible powers. This artifact could be used by gods and mortals alike to achieve awful things.\\
\textbf{Create Plane (4 PP)} Elyndra left this world into the void beyond, tapping into the basest part of her essence to create a realm for those who shared her ways, the Feywild.\\
\textbf{Create Avatar (4 PP)} There was a flash of eldritch lightning, and a silvery, iridescent figure walked the mortal world.

Destroy costs are typically the same as create costs, but may require additional actions in preparation.  For instance, destroying an untended artifact is simple, but destroying an artifact which is currently in the possession of another God may require additional actions, either RP or via PP (such as defeating or tricking that God).  Destroying planes and avatars will always require additional thought and preparation.  Feel free to confer with the admins regarding your plans and they will give you advice on how to proceed.

\vspace{10pt}
\textbf{Abjure Region (2 PP)} Prevents other gods from affecting an area of your choice for the remainder of the week, unless they physically come to the area.  You must remain at the area in question.  If you leave, the effect immediately ends.\\

\textbf{Claim Domain} Allows Gods to increase their portfolio of domains.\\

\textbf{Ascend God}  A 2 PP action which brings a new player into the game.\\ 




\shadowbox{
\begin{minipage}[htbp]{1.2\linewidth}
\vspace{10pt}
    \textbf{0PP Cantrip} This action can produce a variety of low-level or temporary effects.\\
   \textbf{1PP Nourish}  This action causes the target to thrive and prosper.\\
   \textbf{1PP Command} This causes the target to perform, adopt, or discover a particular type of behavior.\\
    \textbf{1PP Mold (Minor)}  This creates or modifies a small amount of land, water, or similar substance.\\
    \textbf{1PP Harm} The opposite of the Nourish action, this brings calamity to the target.\\
    \textbf{1PP Spawn}  This creates a creature that could present a threat to a PC.\\
    \textbf{1PP Forge (Lesser)}  This creates a powerful magical item of limited power. These artifacts only serve to aid mortals or to affect a small area of the world.\\

   \textbf{2PP Mold (Major)}  This creates or modifies a moderate amount of land, water, or similar substance.\\
    \textbf{2PP Craft (Minor)}  This creates an avatar or a pocket demiplane.\\
    \textbf{2PP Forge (Moderate)}  This creates a powerful magical item of intermediate power. These artifacts have a relatively minor use for gods, or they may affect a moderate portion of the world, or they may be structural artifacts.\\
    \textbf{2PP Fashion} Create a new domains, either passive or active.\\
    \textbf{2PP Beget (Minor)} Brings a new God into existence.\\

    \textbf{3PP Beget (Major)}  This creates a being of divine origin, usually an exarch and rarely a mortal hero.\\
    \textbf{3PP Forge (Greater)}  This creates some inanimate object of divine nature (aka, an artifact) that is capable of influencing mortals and the world in general. These artifacts can produce PP or combat bonuses.\\
    \textbf{3PP Shape} This creates a race of civilized beings.\\

    \textbf{4PP Craft (Major)}  This creates an alternate plane of existence.\\
    
    \textbf{XPP Alter} This modifies an existing thing and turns it into something else. This action usually creates a 1pp savings on a chosen action but comes with stipulations. 
        \vspace{10pt}
    \end{minipage}

}


\end{document}